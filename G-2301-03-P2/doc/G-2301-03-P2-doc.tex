\documentclass{mathnotes}
\usepackage{mathnotes}
\usepackage{listings}

\newcommand{\theauthor}{Enrique Cabrerizo Fernández\and Guillermo Ruiz Álvarez}
\newcommand{\thetitle}{Memoria Práctica 2\\Redes de comunicaciones II}

\title{\thetitle}
\author{\theauthor}
\date{Curso 2013 - 2014\\Universidad Autónoma de Madrid}

\begin{document}

\section{Introducción y descripción}
En este documento se detalla el proceso de implementación de un cliente  \href{http://es.wikipedia.org/wiki/Internet_Relay_Chat}{IRC} a partir del código proporcionado de una interfaz \href{http://www.gtk.org/}{GTK}. Para la elaboración del cliente se han seguido las directrices especificadas en el \href{https://tools.ietf.org/html/rfc2812}{RFC 2812}.

La interfaz gráfica de usuario se compone de varias partes. En primer lugar, el usuario puede introducir sus datos de conexión en los campos de texto y conectarse y desconectarse de un servidor utilizando los botones de conexión y desconexión. En segundo lugar, el usuario puede utilizar el campo de introducción de texto principal para enviar comandos al servidor y mensajes a los usuarios conectados en dicho servidor. Finalmente, el usuario puede utilizar los \textit{checkboxes} para modificar los modos de un canal en el que tenga privilegios.

Toda la información recibida aparecerá en la pantalla principal del la interfaz.

\section{Organización de directorios y ficheros.}
En lo que respecta a la organización de los ficheros, se han añadido nuevos directorios modificando ligeramente la estructura:

El directorio raíz, cuyo nombre es \texttt{G-2301-03-P2} contiene los siguientes directorios:

\begin{itemize}
\item \textbf{bin} contiene los ejecutables creados despues de la compilación y el enlazado.

\item \textbf{doc} contiene la documentación del proyecto en formato PDF.

\item \textbf{includes} contiene los ficheros de cabecera generados para la elaboración de librerias y el programa principal.

\item \textbf{lib} contiene las librerias generadas para realizar los programas principales.

\item \textbf{man} contiene los manuales\footnote{Estos manuales han sido generados con doxydoc de doxygen} de las funciones de las librerías. Para acceder a ellos basta ejecutarlos con el programa \texttt{man}.

\item \textbf{obj} contiene los ficheros objeto generados en la compilación.

\item \textbf{src} contiene los ficheros fuente que generarán un ejecutable. Estos son programas de prueba (tests) o los programas principales.

\item \textbf{srclib} contiene los ficheros fuente que generarán los objetos de las librerías.
\end{itemize}

En el directorio raíz se encuentran el fichero Makefile necesario para la compilación y enlazado de los archivos además del fichero de configuración de Doxygen.

\section{Makefile}
Para compilar y enlazar se proporciona un fichero Makefile a ejecutar con el programa \texttt{make}. En esta práctica se han añadido directivas que se presentan a continuación.

\subsection{Directivas}
Las directivas proporcionadas son las siguientes:

\begin{itemize}
\item \textbf{all} genera todos los binarios en el directorio bin\footnote{incluyendo los ficheros objeto necesarios para ello en el directorio obj y las librerias en el directorio lib}. Estos ficheros son:

\subitem G-2301-03-P1-main
\subitem G-2301-03-P2-chat

y todos los binarios creados para realizar tests.

\item \textbf{IRC-SERVER} crea el ejecutable del servidor de la primera práctica en el directorio bin.

\item \textbf{IRC-CLIENT} crea el ejecutable del cliente de la segunda práctica en el directorio bin.

\item \textbf{bin/*} donde * es el nombre de un ejecutable. En este caso se compilará todo lo necesario para compilar y enlazar para la obtención del ejecutable.

\item \textbf{obj/*.o} donde * es el nombre de un objeto. En este caso se compilará lo necesario para compilar el objeto.

\item \textbf{compress}, \textbf{pack}, \textbf{tar}, \textbf{tgz}. Cualquiera de estas directivas ejecutan el comando clean y comprimen la práctica a un fichero tgz que será colocado en el directorio padre del raíz de la práctica.

\item \textbf{clean}, \textbf{clear}. Cualquiera de estas directivas eliminan el fichero tgz, los objeto y los ejecutables.

\item\textbf{G-2301-03-P2-*.a} genera la librería \texttt{G-2301-03-P2-*.a}.
las librerias creadas son

\subitem G-2301-03-P2-libconnection.a
\subitem G-2301-03-P2-libirc.a
\subitem G-2301-03-P2-libaudio.a
\end{itemize}

\subsection{Construcción del Makefile}
El archivo Makefile (los comandos utilizados son de \href{https://www.gnu.org/software/make/manual/make.html#Top}{GNU Make}) se ha realizado siguiendo las siguientes reglas, de forma que puede ser utilizado por cualquier usuario que siga las mismas:
\begin{itemize}
\item Seguir la organización de directorios especificada.
\item No existen ficheros de cabecera para los ficheros fuente de src.
\item Todos los ficheros de srclib tienen un fichero de cabecera asociado.
\end{itemize}
Para cambiar el nombre de los directorios o añadir ficheros de cabecera extra, basta con modificar las macros al incio del fichero Makefile.

\section{Funciones de librería}
En esta práctica se ha ampliado el número de librerías ofreciendo tres librerías diferenciadas según la utilidad de las mismas:

\begin{itemize}
\item \textbf{G-2301-03-P2-libconnection.a} ofrece todas las funciones necesarias de conexión, incluyendo el manejo de hilos, la daemonización y el uso de semáforos.

\item \textbf{G-2301-03-P2-libirc.a} ofrece todas las funciones necesarias relacionadas con IRC. Incluyendo funciones de cliente y servidor.

\item \textbf{G-2301-03-P2-libaudio.a} ofrece todas las funciones necesarias relacionadas con las llamadas y el manejo de audio.
\end{itemize}

La descripción precisa de todas estas funciones se encuentra en la páginas de manual (\texttt{man pages}) realizadas por los autores en el directorio correspondiente.

\section{Estructura, funcionalidad y pruebas del cliente}

El cliente implementado parte de la implementación de la interfaz gráfica en el fichero \textbf{G-2301-P2-chat.c}. Todos los eventos de dicha interfaz producen la ejecución de las funciones implementadas en \textbf{G-2301-P2-chat$\_$funcs.c}. Se ha añadido un módulo auxiliar llamado \textbf{G-2301-P2-client$\_$utility$\_$functions.c} donde se implementa el hilo principal que recibe datos del servidor y las funciones necesarias para el envío de comandos y tratamiento de los datos recibidos.

El proceso y estructura de la implementación de la conexión, desconexión y el envío y recibo de comandos se describe a continuación.

\subsection{Conexión y desconexión}
Cuando el usuario desee conectarse a un servidor IRC es \textbf{necesario} que rellene la totalidad de los campos de información de usuario (Apodo, Nombre, Nombre real, Servidor\footnote{Puede utilizarse tanto la IP con la URL del servidor de chat.}). En el caso de que el campo de puerto se deje vacío o se escriba IRC se utilizará el puerto predeterminado de IRC 6667. En el momento en el que se pulsa el botón de conexión se ejecuta una función implementada en \textbf{G-2301-P2-chat$\_$funcs.c} que se encarga de manejar el evento. Dicha función realiza llamada a una función del fichero \textbf{G-2301-P2-client$\_$utility$\_$functions.c} que se encarga de realizar las acciones necesarias para establecer una conexión con el servidor.

En el caso de la desconexión, el evento se captura y se manda una petición de desconexión al servidor. Una vez recibida la respuesta, el cliente se desconecta.

\section{Ampliación multimedia del cliente}
Se compone la cabecera RTP a partir de dos funciones, rtp_header_setup que fija una serie de parametros que van a ser fijos durante toda la conexion y rtp_header_build que se llama para completar la cabecera con los parámetros que varían para cada paquete (timestamp y numero de secuencia).

\end{document}